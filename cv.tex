\documentclass{resume} % Use the custom template.cls style
\usepackage[left=0.4 in,top=0.4in,right=0.4 in,bottom=0.4in]{geometry} % Document margins
\newcommand{\tab}[1]{\hspace{.2667\textwidth}\rlap{#1}} 
\newcommand{\itab}[1]{\hspace{0em}\rlap{#1}}
\name{Pan Chen}
\address{+1(647) 686-1520 \\ Toronto, Ontario, CANADA} 
\address{\href{mailto:panchen@cs.toronto.edu}{panchen@cs.toronto.edu} \\ \href{https://linkedin.com/in/chenpanxyz}{linkedin.com/in/chenpanxyz} \\ \href{https://www.cs.toronto.edu/~panchen}{chenpan.ca} \\ @chenpanxyz}%
\begin{document}

%%%%% Education
\begin{rSection}{Education}

    \textbf{PhD in Computer Science}, University of Toronto \hfill {2023 - Present}
    \begin{itemize}
        \item Research interests: Computational Creativity, AI in Scientific Discoveries, Human-Computer Interaction
        \item Supervisor: Professor Alán Aspuru-Guzik
    \end{itemize}
    
    \textbf{Summer School}, Carnegie Mellon University \hfill {July 2023}
    
    \textbf{Bachelor of Science in Computer Science and Statistics}, University of Toronto \hfill {2018 - 2022}
    \begin{itemize}
        \item Undergraduate research advisor: Professor Joseph Jay Williams
        \item Highlighted courses: Neural Networks and Deep Learning, Applied Bayesian Statistics, Statistical Methods for Machine Learning, Analysis \& Complexity, Image Understanding
    \end{itemize}

\end{rSection}

%%%%% Experience
\begin{rSection}{EXPERIENCE}
    \textbf{Research Assistant} \hfill September 2022 - December 2023\\DGP Lab, University of Toronto \hfill \textit{Toronto, Ontario}
    \begin{itemize}
        \itemsep -3pt {} 
        \item Member of the team that won the XPRIZE Digital Learning Challenge by using applied machine learning algorithms to improve the learning experience
    \end{itemize}

    \textbf{Research Scholar} \hfill May 2022 - August 2022\\Data Sciences Institute \hfill \textit{Toronto, Ontario}
    \begin{itemize}
        \itemsep -3pt {} 
        \item Developed a platform for people to collect data from third-party websites
        \item Prepared different data analysis scripts for people to run in the front end
    \end{itemize}

 
    \textbf{Teaching Assistant} \hfill Sep 2021 - Present\\University of Toronto \hfill \textit{Toronto, Ontario}
    \begin{itemize}
        \itemsep -3pt {} 
        \item Appointments in \textbf{2} departments (CS \& Stats)
    \end{itemize}
    
    \textbf{Software Developer Co-op} \hfill Jun 2020 - Jun 2021\\Infrastructures for Information \hfill \textit{Toronto, Ontario}
    \begin{itemize}
        \itemsep -3pt {} 
        \item Software developer for a tool in Java that boosts productivity when working on multiple labelling documents
    \end{itemize}

\end{rSection} 

%%%%% Projects

\begin{rSection}{PROJECTS}
\vspace{-1.25em}
\item \textbf{AIEPlatform} {A modern platform for data-driven experiments that adapt and improve over time by utilizing machine learning algorithms.}

\item \textbf{Digital Learning Challenge by XPrize} {Lead software developer \& Machine Learning designer for the cross-platform infrastructure that supports both traditional and adaptive experiments and Machine Learning. We collaborated with CMU \& UNC, and deployed our infrastructure in more than \textbf{25} courses. We are the grand winner of this XPRIZE Digital Learning Challenge sponsored by IES. \href{https://www.xprize.org/challenge/digitallearning/finalist-teams}{(More)}}

\item \textbf{Voice Reflection System} {Lead designer \& developer of this online reflection system that allows students to reflect on course topics by talking. This system has been used by more than 500 students at the University of Toronto.}

\item \textbf{Face-Control Snake Game} {Applied a TensorFlow model called PoseNet to detect the player's position in real-time, so that the player can move their face to control the snake. \href{https://www.youtube.com/watch?v=Zn_5Oy7DZJw}{(Video)}}
\end{rSection} 
\end{document}