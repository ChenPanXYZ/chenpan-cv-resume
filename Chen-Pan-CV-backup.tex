\documentclass{resume} % Use the custom template.cls style
\usepackage[left=0.4in,top=0.4in,right=0.4in,bottom=0.4in]{geometry} % Document margins
\usepackage{graphicx} % Package to include images
\usepackage{lipsum} % For dummy text, can be removed
\newcommand{\tab}[1]{\hspace{.2667\textwidth}\rlap{#1}} 
\newcommand{\itab}[1]{\hspace{0em}\rlap{#1}}
\name{Pan Chen}
\address{+1(647) 686-1520 \\ Toronto, Ontario, CANADA} 
\address{\href{mailto:panchen@cs.toronto.edu}{panchen@cs.toronto.edu} \\ \href{https://linkedin.com/in/chenpanxyz}{linkedin.com/in/chenpanxyz} \\ \href{https://www.cs.toronto.edu/~panchen}{panchen.ca}%

\begin{document}

% Adding the QR code image to the top right corner
\begin{picture}(0,0)
    \put(0,20){\includegraphics[width=2cm]{qrcode.png}}
\end{picture}

%%%%% Research Interests
\textit{PhD student in Computer Science at the University of Toronto specializing in applied machine learning, intelligent systems, and human-computer interaction. Strong background in systems development, algorithm design, and research communication. Published and presented work in JMIR, SIGCITE, ICML, and XPrize Digital Learning Challenge. Passionate about robotics, autonomy, and scientific discovery through AI.}

% \textit{\textbf{Expert in Python, C, and Java, with experience in machine learning, optimization, and large-scale system development for adaptive learning and scientific computing applications.}}
% \begin{rSection}{Research Interests}
%     AI for Science (scientific discoveries), LLM, Computational Creativity, Human-Computer Interaction
% \end{rSection}

%%%%% Education
\begin{rSection}{Education}

    \textbf{PhD Student in Computer Science}, University of Toronto \hfill {2023 - Present}
    \begin{itemize}
        \item Supervisor: Dr. Alán Aspuru-Guzik
        \item Regular Committee Members: Dr. Nicholas Papernot, Dr. Michael Liut
    \end{itemize}
    
    \textbf{Summer School}, Carnegie Mellon University \hfill {Jul 2023}
    \begin{itemize}
        \item Mentors: Dr. John Stamper, Dr. Steven Moore
    \end{itemize}
    
    \textbf{Bachelor of Science in Computer Science and Statistics}, University of Toronto \hfill {2018 - 2022}
    \begin{itemize}
        \item Research advisor: Dr. Joseph Jay Williams
    \end{itemize}

\end{rSection}

\begin{rSection}{Technical Skills}
\begin{tabular}{ l l }
\textbf{Programming Languages:} & Python, C, Java, Bash, SQL, NoSQL \\
\textbf{Frameworks / Libraries:} & PyTorch, TensorFlow, OpenCV, scikit-learn, NumPy, Pandas \\
\textbf{Tools:} & Git, Docker, Linux, Jupyter, LaTeX \\
\textbf{Systems / Platforms:} & Scientific Computing, Distributed Systems, Web Infrastructure \\
\textbf{Other:} & Experimental Design, Academic Writing
\end{tabular}
\end{rSection}

%%%%% Experience
\begin{rSection}{Experience}
    \textbf{Instructor} \hfill Jan 2025 - Present\\Departments of Computer Science\hfill \textit{Toronto, Ontario}
    \begin{itemize}
        \itemsep -3pt {} 
        \item CSC148, Introduction to Computer Science (Python)
        \item CSC207, Software Design (Java)
    \end{itemize}
    \textbf{Research Assistant} \hfill Sep 2022 - Dec 2023\\Dynamic Graphics Project Lab \& Data Sciences Institute, University of Toronto \hfill \textit{Toronto, Ontario}
    \begin{itemize}
    \end{itemize}

    % \textbf{Research Scholar} \hfill May 2022 - Aug 2022\\Data Sciences Institute \hfill \textit{Toronto, Ontario}
    % \begin{itemize}
    %     \itemsep -3pt {} 
    %     \item Developed a platform for people to collect data from third-party websites
    %     \item Prepared different data analysis scripts for people to run in the front end
    % \end{itemize}

    \textbf{Teaching Assistant} \hfill Sep 2021 - Dec 2024\\Departments of Computer Science and Statistics, University of Toronto \hfill \textit{Toronto, Ontario}
    
    \textbf{Software Developer Co-op} \hfill Jun 2020 - Jun 2021\\Infrastructures for Information \hfill \textit{Toronto, Ontario}

\end{rSection} 

%%%%% Projects
\begin{rSection}{PROJECTS}
\vspace{-1.25em}
    \item \textbf{Schema-Based In-Context Learning}
    \begin{itemize}
        \item Led a cross-functional team of machine learning and cognitive science researchers.
        \item Developed a schema-based in-context learning framework to enhance scientific discovery.
        \item Utilized transformer-based models to represent and generalize scientific abstractions.
        \item Achieved \textbf{15\% improvement} in chemistry tasks (in progress).
    \end{itemize}
    \newpage
    \item \textbf{Computer Vision for Materials Synthesis}
    \begin{itemize}
        \item Contributed to the development and evaluation of a computer vision model for \textbf{robotic materials synthesis}.
        \item Achieved \textbf{83\% accuracy} in phase labeling, outperforming human annotation benchmarks.
        \item Collaborated with researchers in materials science and machine learning.
    \end{itemize}

    \item \textbf{Adaptive Experiment Infrastructure for Science}
    \begin{itemize}
        \item Built a modern platform for data-driven experiments.
        \item Designed the system to adapt and improve over time using machine learning algorithms.
    \end{itemize}

    \item \textbf{Digital Learning Challenge by XPrize}
    \begin{itemize}
        \item Lead software developer \& machine learning designer for cross-platform machine learning infrastructure.
        \item Collaborated with \textbf{CMU \& UNC} and deployed infrastructure in over \textbf{25 courses}.
        \item \textbf{Grand Winner} of the XPRIZE Digital Learning Challenge, sponsored by IES. \href{https://www.xprize.org/challenge/digitallearning/finalist-teams}{More}
    \end{itemize}

    \item \textbf{Voice Reflection System}
    \begin{itemize}
        \item Lead designer \& developer of an online voice-based reflection system.
        \item Enables 500+ students to reflect on course topics by speaking at the University of Toronto.
    \end{itemize}

% \item \textbf{Face-Control Snake Game} {Applied a TensorFlow model called PoseNet to detect the player's position in real-time, so that the player can move their face to control the snake. \href{https://www.youtube.com/watch?v=Zn_5Oy7DZJw}{(Video)}}
\end{rSection} 

\begin{rSection}{PUBLICATIONS}
\item Chen, P., et al. (n.d.). Schema-Based Reasoning: A new paradigm for in-context learning (Work in progress).
\item Zhang, Z., Chen, P., Du, F., Ye, R., Huang, O., Liut, M., & Aspuru-Guzik, A. (n.d.). TreeReader: A hierarchical academic paper reader powered by language models (Manuscript under review).
\item Gaidimas, M. A., Mandal, A., Chen, P., Leong, S. X., Kim, G.-H., Talekar, A., Kirlikovali, K. O., Darvish, K., Farha, O. K., Bernales, V., & Aspuru-Guzik, A. (n.d.). Computer vision for high-throughput materials synthesis: A tutorial for experimentalists (Manuscript under review).
\item Chen, P., Zavaleta Bernuy, A., Liut, M., & Williams, J. J. (2024). Adaptive experiments for continuous improvement in computer science education: A case study. In Proceedings of the 26th Western Canadian Conference on Computing Education (pp. 1-7).
\item Bhattacharjee, A., Chen, P., Mandal, A., Hsu, A., O'Leary, K., Mariakakis, A., & Williams, J. J. (2024). Exploring user perspectives on brief reflective questioning activities for stress management: Mixed methods study. JMIR Formative Research, 8(1), e47360.
\item Ye, R., Chen, P., Mao, Y., Wang-Lin, A., Shaikh, H., Zavaleta Bernuy, A., & Williams, J. J. (2022, September). Behavioral consequences of reminder emails on students’ academic performance: A real-world deployment. In Proceedings of the 23rd Annual Conference on Information Technology Education (pp. 16-22).
\item Musabirov, I., Zavaleta Bernuy, A., Chen, P., Liut, M., & Williams, J. (2024, May). Opportunities for Adaptive Experiments to Enable Continuous Improvement in Computer Science Education. In Proceedings of the 26th Western Canadian Conference on Computing Education (pp. 1-7).
\item Chen, P., Sibia, N., Zavaleta Bernuy, A., Liut, M., & Williams, J. J. (2022, March). Investigating the Impact of Voice Response Options in Surveys. In Proceedings of the 53rd ACM Technical Symposium on Computer Science Education V. 2 (pp. 1124-1124).
\item Han, Z., Gorobets, E., & Chen, P. (2022). Parameter efficient dendritic-tree neurons outperform perceptrons. arXiv preprint arXiv:2207.00708. Work presented at ICML 2022 Dynamic Neural Networks Workshop.
\end{rSection}
\vfill
\begin{rSection}{Additional Information}
\textit{\textbf{Country of Permanent Residence:}} Canada
\end{document}
