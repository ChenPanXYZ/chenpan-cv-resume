\documentclass{resume} % Use the custom template.cls style
\usepackage[left=0.4in,top=0.4in,right=0.4in,bottom=0.4in]{geometry} % Document margins
\usepackage{graphicx} % Package to include images
\usepackage{lipsum} % For dummy text, can be removed
\newcommand{\tab}[1]{\hspace{.2667\textwidth}\rlap{#1}} 
\newcommand{\itab}[1]{\hspace{0em}\rlap{#1}}
\name{Pan Chen}
\address{+1(647) 686-1520 \\ Toronto, Ontario, CANADA \\ Permanent Resident of Canada}
\address{\href{mailto:panchen@cs.toronto.edu}{panchen@cs.toronto.edu} \\ \href{https://linkedin.com/in/chenpanxyz}{linkedin.com/in/chenpanxyz} \\ \href{https://www.panchen.ca}{panchen.ca}}%

\begin{document}

% Adding the QR code image to the top right corner
\begin{picture}(0,0)
    \put(0,20){\includegraphics[width=1.8cm]{qrcode.png}}
\end{picture}

%%%%% Research Interests
\begin{rSection}{Research Interests}
    AI for Science (scientific discoveries), LLM, Computational Creativity, Human-Computer Interaction
\end{rSection}

%%%%% Education
\begin{rSection}{Education}

    \textbf{PhD Student in Computer Science}, University of Toronto \hfill {2022 - Present}
    \begin{itemize}
        \item Supervisor: Dr. Alán Aspuru-Guzik
        \item Regular Committee Members: Dr. Nicholas Papernot, Dr. Michael Liut
    \end{itemize}
    
    \textbf{Summer School (Full Scholarship)}, Carnegie Mellon University \hfill {Jul 2023}
    \begin{itemize}
        \item Mentors and Paper Collaborators: Dr. John Stamper, Dr. Steven Moore
    \end{itemize}
    
    \textbf{Bachelor of Science in Computer Science and Statistics}, University of Toronto \hfill {2018 - 2022}
    \begin{itemize}
        \item Research advisor: Dr. Joseph Jay Williams
    \end{itemize}

\end{rSection}

%%%%% Experience
\begin{rSection}{Experience}
    \textbf{Instructor} \hfill Jan 2025 - Present\\Departments of Computer Science\hfill \textit{Toronto, Ontario}
    \begin{itemize}
        \itemsep -3pt {} 
        \item CSC148 (Introduction to Computer Science), CSC207 (Software Design), CSC309 (Programming on the Web)
        \item Teaching students foundational computer science concepts and practical skills that will empower them in their academic and professional journeys.
    \end{itemize}
    \textbf{Research Assistant} \hfill Sep 2022 - Dec 2023\\Dynamic Graphics Project Lab, University of Toronto \hfill \textit{Toronto, Ontario}
    \begin{itemize}
        \itemsep -3pt {} 
        \item Led team and win the \href{https://www.entrepreneurship.artsci.utoronto.ca/news/cs-graduates-and-professor-awarded-grand-prize-xprize-digital-learning-challenge}{XPRIZE Digital Learning Challenge} by using applied machine learning algorithms to improve the learning experience 
        \item Gained significant media attention, being reported by different news sources
    \end{itemize}

    \textbf{Research Scholar} \hfill May 2022 - Aug 2022\\Data Sciences Institute (DSI), advisor:  Dr. Michael Liut \hfill \textit{Toronto, Ontario}
    \begin{itemize}
        \itemsep -3pt {} 
        \item Developed a platform for people to collect data from third-party websites
        \item Prepared different data analysis scripts for people to run in the front end
    \end{itemize}

    \textbf{Teaching Assistant} \hfill Sep 2021 - Dec 2024\\Departments of Computer Science and Statistics, University of Toronto \hfill \textit{Toronto, Ontario}
        \begin{itemize}
        \itemsep -3pt {} 
        \item Worked with different faculty to reinnovate course materials and improve student learning experience
    \end{itemize}
    \textbf{Software Developer Co-op @ Infrastructures for Information (i4i)} \hfill Jun 2020 - Jun 2021\\ \hfill \textit{Toronto, Ontario}
        \begin{itemize}
        \itemsep -3pt {} 
        \item Developed a platform for people to collect data from third-party websites
        \item Prepared different data analysis scripts for people to run in the front end
    \end{itemize}

\end{rSection}
\newpage 
\begin{rSection}{Publications}
\vspace{-1.25em}
\item Chen, P., et al. (n.d.). Schema-Based Reasoning: A new paradigm for in-context learning (Paper submitted to ICLR 2026).
\item Zhang, Z., Chen, P., Du, F., Ye, R., Huang, O., Liut, M., & Aspuru-Guzik, A. (2025). TreeReader: A Hierarchical Academic Paper Reader Powered by Language Models. arXiv preprint arXiv:2507.18945. (Paper accepted by VL/HCC 2025)
\item Gaidimas, M., Mandal, A., Chen, P., Leong, S. X., Kim, G. H., Talekar, A., ... & Aspuru-Guzik, A. (2025). Computer Vision for High-Throughput Materials Synthesis: A Tutorial for Experimentalists.
\item Chen, P., Zavaleta Bernuy, A., Liut, M., & Williams, J. J. (2024). Adaptive experiments for continuous improvement in computer science education: A case study. In Proceedings of the 26th Western Canadian Conference on Computing Education (pp. 1-7).
\item Bhattacharjee, A., Chen, P., Mandal, A., Hsu, A., O'Leary, K., Mariakakis, A., & Williams, J. J. (2024). Exploring user perspectives on brief reflective questioning activities for stress management: Mixed methods study. JMIR Formative Research, 8(1), e47360.
\item Ye, R., Chen, P., Mao, Y., Wang-Lin, A., Shaikh, H., Zavaleta Bernuy, A., & Williams, J. J. (2022, September). Behavioral consequences of reminder emails on students’ academic performance: A real-world deployment. In Proceedings of the 23rd Annual Conference on Information Technology Education (pp. 16-22).
\item Musabirov, I., Zavaleta Bernuy, A., Chen, P., Liut, M., & Williams, J. (2024, May). Opportunities for Adaptive Experiments to Enable Continuous Improvement in Computer Science Education. In Proceedings of the 26th Western Canadian Conference on Computing Education (pp. 1-7).
\item Chen, P., Sibia, N., Zavaleta Bernuy, A., Liut, M., & Williams, J. J. (2022, March). Investigating the Impact of Voice Response Options in Surveys. In Proceedings of the 53rd ACM Technical Symposium on Computer Science Education V. 2 (pp. 1124-1124).
\item Han, Z., Gorobets, E., & Chen, P. (2022). Parameter efficient dendritic-tree neurons outperform perceptrons. arXiv preprint arXiv:2207.00708. Work presented at ICML 2022 Dynamic Neural Networks Workshop.
\end{rSection}
%%%%% Projects
\begin{rSection}{PROJECTS}
\vspace{-1.25em}
\item \textbf{Schema-Based In-Context Learning} {Mimic the human ability to learn from abstractions to enhance scientific understanding.}
\item \textbf{Adaptive Experiments for Science} {A modern platform for data-driven experiments that adapt and improve over time by utilizing machine learning algorithms.}

\item \textbf{Digital Learning Challenge by XPrize} {Lead software developer \& Machine Learning designer for the cross-platform infrastructure that supports both traditional and adaptive experiments and Machine Learning. We collaborated with CMU \& UNC, and deployed our infrastructure in more than \textbf{25} courses. We are the grand winner of this XPRIZE Digital Learning Challenge sponsored by IES. \href{https://www.xprize.org/challenge/digitallearning/finalist-teams}{(More)}}

\item \textbf{Voice Reflection System} {Lead designer \& developer of this online reflection system that allows students to reflect on course topics by talking. This system has been used by more than 500 students at the University of Toronto.}

\item \textbf{Face-Control Snake Game} {Applied a TensorFlow model called PoseNet to detect the player's position in real-time, so that the player can move their face to control the snake. \href{https://www.youtube.com/watch?v=Zn_5Oy7DZJw}{(Video)}}
\end{rSection} 

% \begin{rSection}{Personal Information}
%     Citizenship/Permanent Residence: Canada, China
    
% \end{rSection}

\end{document}
